\documentclass[aspectratio=169]{beamer}
\usepackage[orientation=landscape,size=custom,width=16,height=9,scale=0.5,debug]{beamerposter}
\usepackage[T2A]{fontenc}
\usepackage[utf8]{inputenc}
\usepackage[english,russian]{babel}
\usepackage{cite,enumerate,float,indentfirst}
\usepackage{graphicx}
\usepackage{verbatim}
\usepackage{listings}
\lstset{language=Python}
\usetheme{umbc2}

\title{Типовые задачи в разработке ПО и их решения}
\author{Alexey Shrub}
\date{2012-09-13}
\begin{document}

\maketitle

\begin{frame}{Методологии разработки}
\begin{itemize}
\item ООП
\pause
\item Функциональный стиль
\end{itemize}
\end{frame}

\begin{frame}{Конфигурирование}
\begin{itemize}
\item Зачем?
\pause
\item Файлы или база?
\pause
\item Форматы файлов и библиотеки.
\end{itemize}
\end{frame}

\begin{frame}{Пример YAML конфига}
\verbatiminput{examples/conf.yml}
%\lstinputlisting{examples/conf.yml}
\end{frame}

\begin{frame}{Журналирование}
\begin{itemize}
\item Зачем?
\pause
\item Что? (Уровни)
\pause
\item Куда? (БД или файлы)
\end{itemize}
\end{frame}

\begin{frame}{Профилирование}
\begin{itemize}
\item Зачем?
\pause
\item Инструменты.
\end{itemize}
\end{frame}

\begin{frame}{Протоколы интеграции и маршалинг}
\begin{itemize}
\item Интеграция: файлы, базы, очереди, вызов процедур
\pause
\item Вызовы процедур: SOAP, XML::RPC, JSON::RPC, в какой-то мере REST
\pause
\item Сериализация/Маршалинг: JSON, YAML
\end{itemize}
\end{frame}

\begin{frame}{Хранилища данных}
\begin{itemize}
\item РСУБД (PostgreSQL)
\pause
\item Ключ-значение (Redis, Memcached, Riak)
\pause
\item Документоориентированные (mongoDB)
\end{itemize}
\end{frame}

\begin{frame}{Технологии параллелизма}
\begin{itemize}
\item process
\pause
\item thread
\pause
\item event loop
\end{itemize}
\end{frame}

\begin{frame}{Масштабирование ПО}
\begin{itemize}
\item Один в поле не воин
\pause
\item Как распределить запросы (с сессиями и без них)
\pause
\item Репликация. Партицирование
\end{itemize}
\end{frame}

\begin{frame}{Сложность алгоритмов}
\begin{itemize}
\item Временная сложность алгоритма (в худшем случае)
\begin{itemize}
\item $O(log \: n)$ - бинарный поиск по отсортированному массиву
\item $O(n)$ - поиск в одномерном массиве полным перебором
\item $O(n^2)$ - поиск в двумерном массиве полным перебором, многие виды сортировок
\item $O(c^n)$ - решение задачи коммивояжёра методами динамического программирования
\item $O(n!)$ - решение задачи коммивояжёра полным перебором
\end{itemize}
\pause
\item В лучшем и средних случаях
\pause
\item Пространственная сложность - по раходу памяти
\end{itemize}
\end{frame}

\begin{frame}{Оптимизация}
\begin{itemize}
\item Всему своё время
\pause
\item Компилятор умный
\pause
\item Узкие места на нужном языке
\end{itemize}
\end{frame}

\begin{frame}{Установка ПО}
\begin{itemize}
\item Версии
\pause
\item Зависимости
\pause
\item Откат
\end{itemize}
\end{frame}

\begin{frame}{Локализация и интернационализация (технологии и подходы)}
\begin{itemize}
\item Интернационализация - чтоб могло (utf8, gettext, .po файлы и инструменты для них)
\pause
\item Локализация - для конкретной страны (перевод, форматы дат, денег)
\end{itemize}
\end{frame}

\begin{frame}{Конечные автоматы?}
\begin{itemize}
\item Регулярные выражения
\pause
\item PCRE
\end{itemize}
\end{frame}

\begin{frame}{"Читай много, но не очень много книг." Бенджамин Франклин}
\begin{itemize}
\item "Совершенный код" Стив Макконнелл
\pause
\item "Программист-прагматик. Путь от подмастерья к мастеру" Э. Хант, Д. Томас
\pause
\item "Рефакторинг. Улучшение существующего кода" Мартин Фаулер
\pause
\item "Приемы объектно-ориентированного проектирования. Паттерны проектирования" Э. Гамма, Р. Хелм, Р. Джонсон, Дж. Влиссидес
\pause
\item "Этюды для программистов" Чарльз Уэзерелл
\pause
\item "Искусство программирования" Дональд Кнут
\pause
\item "Регулярные выражения" Джеффри Фридл
\end{itemize}
\end{frame}

\begin{frame}{Вопросы?}
\begin{block}{Исходники презентации (LaTeX, Beamer):}
https://github.com/worldmind/typical-dev-tasks-presentation-ru.git
\begin{center}
\includegraphics{qr-git-url.png}
\end{center}
\end{block}
\begin{block}{Feedback to: ashrub@yandex.ru}
\end{block}

\end{frame}

\end {document}

